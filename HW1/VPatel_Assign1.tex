\documentclass[]{article}
\usepackage{lmodern}
\usepackage{amssymb,amsmath}
\usepackage{ifxetex,ifluatex}
\usepackage{fixltx2e} % provides \textsubscript
\ifnum 0\ifxetex 1\fi\ifluatex 1\fi=0 % if pdftex
  \usepackage[T1]{fontenc}
  \usepackage[utf8]{inputenc}
\else % if luatex or xelatex
  \ifxetex
    \usepackage{mathspec}
  \else
    \usepackage{fontspec}
  \fi
  \defaultfontfeatures{Ligatures=TeX,Scale=MatchLowercase}
\fi
% use upquote if available, for straight quotes in verbatim environments
\IfFileExists{upquote.sty}{\usepackage{upquote}}{}
% use microtype if available
\IfFileExists{microtype.sty}{%
\usepackage{microtype}
\UseMicrotypeSet[protrusion]{basicmath} % disable protrusion for tt fonts
}{}
\usepackage[margin=1in]{geometry}
\usepackage{hyperref}
\hypersetup{unicode=true,
            pdftitle={DATA 621},
            pdfauthor={Charls Joseph, Mary Anna Kivenson, Elina Azrilyan, Sunny Mehta, Vinayak Patel},
            pdfborder={0 0 0},
            breaklinks=true}
\urlstyle{same}  % don't use monospace font for urls
\usepackage{color}
\usepackage{fancyvrb}
\newcommand{\VerbBar}{|}
\newcommand{\VERB}{\Verb[commandchars=\\\{\}]}
\DefineVerbatimEnvironment{Highlighting}{Verbatim}{commandchars=\\\{\}}
% Add ',fontsize=\small' for more characters per line
\usepackage{framed}
\definecolor{shadecolor}{RGB}{248,248,248}
\newenvironment{Shaded}{\begin{snugshade}}{\end{snugshade}}
\newcommand{\AlertTok}[1]{\textcolor[rgb]{0.94,0.16,0.16}{#1}}
\newcommand{\AnnotationTok}[1]{\textcolor[rgb]{0.56,0.35,0.01}{\textbf{\textit{#1}}}}
\newcommand{\AttributeTok}[1]{\textcolor[rgb]{0.77,0.63,0.00}{#1}}
\newcommand{\BaseNTok}[1]{\textcolor[rgb]{0.00,0.00,0.81}{#1}}
\newcommand{\BuiltInTok}[1]{#1}
\newcommand{\CharTok}[1]{\textcolor[rgb]{0.31,0.60,0.02}{#1}}
\newcommand{\CommentTok}[1]{\textcolor[rgb]{0.56,0.35,0.01}{\textit{#1}}}
\newcommand{\CommentVarTok}[1]{\textcolor[rgb]{0.56,0.35,0.01}{\textbf{\textit{#1}}}}
\newcommand{\ConstantTok}[1]{\textcolor[rgb]{0.00,0.00,0.00}{#1}}
\newcommand{\ControlFlowTok}[1]{\textcolor[rgb]{0.13,0.29,0.53}{\textbf{#1}}}
\newcommand{\DataTypeTok}[1]{\textcolor[rgb]{0.13,0.29,0.53}{#1}}
\newcommand{\DecValTok}[1]{\textcolor[rgb]{0.00,0.00,0.81}{#1}}
\newcommand{\DocumentationTok}[1]{\textcolor[rgb]{0.56,0.35,0.01}{\textbf{\textit{#1}}}}
\newcommand{\ErrorTok}[1]{\textcolor[rgb]{0.64,0.00,0.00}{\textbf{#1}}}
\newcommand{\ExtensionTok}[1]{#1}
\newcommand{\FloatTok}[1]{\textcolor[rgb]{0.00,0.00,0.81}{#1}}
\newcommand{\FunctionTok}[1]{\textcolor[rgb]{0.00,0.00,0.00}{#1}}
\newcommand{\ImportTok}[1]{#1}
\newcommand{\InformationTok}[1]{\textcolor[rgb]{0.56,0.35,0.01}{\textbf{\textit{#1}}}}
\newcommand{\KeywordTok}[1]{\textcolor[rgb]{0.13,0.29,0.53}{\textbf{#1}}}
\newcommand{\NormalTok}[1]{#1}
\newcommand{\OperatorTok}[1]{\textcolor[rgb]{0.81,0.36,0.00}{\textbf{#1}}}
\newcommand{\OtherTok}[1]{\textcolor[rgb]{0.56,0.35,0.01}{#1}}
\newcommand{\PreprocessorTok}[1]{\textcolor[rgb]{0.56,0.35,0.01}{\textit{#1}}}
\newcommand{\RegionMarkerTok}[1]{#1}
\newcommand{\SpecialCharTok}[1]{\textcolor[rgb]{0.00,0.00,0.00}{#1}}
\newcommand{\SpecialStringTok}[1]{\textcolor[rgb]{0.31,0.60,0.02}{#1}}
\newcommand{\StringTok}[1]{\textcolor[rgb]{0.31,0.60,0.02}{#1}}
\newcommand{\VariableTok}[1]{\textcolor[rgb]{0.00,0.00,0.00}{#1}}
\newcommand{\VerbatimStringTok}[1]{\textcolor[rgb]{0.31,0.60,0.02}{#1}}
\newcommand{\WarningTok}[1]{\textcolor[rgb]{0.56,0.35,0.01}{\textbf{\textit{#1}}}}
\usepackage{longtable,booktabs}
\usepackage{graphicx}
% grffile has become a legacy package: https://ctan.org/pkg/grffile
\IfFileExists{grffile.sty}{%
\usepackage{grffile}
}{}
\makeatletter
\def\maxwidth{\ifdim\Gin@nat@width>\linewidth\linewidth\else\Gin@nat@width\fi}
\def\maxheight{\ifdim\Gin@nat@height>\textheight\textheight\else\Gin@nat@height\fi}
\makeatother
% Scale images if necessary, so that they will not overflow the page
% margins by default, and it is still possible to overwrite the defaults
% using explicit options in \includegraphics[width, height, ...]{}
\setkeys{Gin}{width=\maxwidth,height=\maxheight,keepaspectratio}
\IfFileExists{parskip.sty}{%
\usepackage{parskip}
}{% else
\setlength{\parindent}{0pt}
\setlength{\parskip}{6pt plus 2pt minus 1pt}
}
\setlength{\emergencystretch}{3em}  % prevent overfull lines
\providecommand{\tightlist}{%
  \setlength{\itemsep}{0pt}\setlength{\parskip}{0pt}}
\setcounter{secnumdepth}{0}
% Redefines (sub)paragraphs to behave more like sections
\ifx\paragraph\undefined\else
\let\oldparagraph\paragraph
\renewcommand{\paragraph}[1]{\oldparagraph{#1}\mbox{}}
\fi
\ifx\subparagraph\undefined\else
\let\oldsubparagraph\subparagraph
\renewcommand{\subparagraph}[1]{\oldsubparagraph{#1}\mbox{}}
\fi

%%% Use protect on footnotes to avoid problems with footnotes in titles
\let\rmarkdownfootnote\footnote%
\def\footnote{\protect\rmarkdownfootnote}

%%% Change title format to be more compact
\usepackage{titling}

% Create subtitle command for use in maketitle
\providecommand{\subtitle}[1]{
  \posttitle{
    \begin{center}\large#1\end{center}
    }
}

\setlength{\droptitle}{-2em}

  \title{DATA 621}
    \pretitle{\vspace{\droptitle}\centering\huge}
  \posttitle{\par}
    \author{Charls Joseph, Mary Anna Kivenson, Elina Azrilyan, Sunny Mehta, Vinayak
Patel}
    \preauthor{\centering\large\emph}
  \postauthor{\par}
      \predate{\centering\large\emph}
  \postdate{\par}
    \date{2020-02-29}


\begin{document}
\maketitle

\hypertarget{homework-1}{%
\section{HOMEWORK \#1}\label{homework-1}}

\hypertarget{overview}{%
\subsection{Overview:}\label{overview}}

In this homework assignment, you will explore, analyze and model a data
set containing approximately 2200 records. Each record represents a
professional baseball team from the years 1871 to 2006 inclusive. Each
record has the performance of the team for the given year, with all of
the statistics adjusted to match the performance of a 162 game season.
Your objective is to build a multiple linear regression model on the
training data to predict the number of wins for the team. You can only
use the variables given to you (or variables that you derive from the
variables provided).

\hypertarget{deliverables}{%
\subsection{Deliverables:}\label{deliverables}}

\begin{itemize}
\tightlist
\item
  A write-up submitted in PDF format. Your write-up should have four
  sections. Each one is described below. You may assume you are
  addressing me as a fellow data scientist, so do not need to shy away
  from technical details.
\item
  Assigned predictions (the number of wins for the team) for the
  evaluation data set.
\item
  Include your R statistical programming code in an Appendix.
\end{itemize}

\begin{enumerate}
\def\labelenumi{\arabic{enumi}.}
\tightlist
\item
  DATA EXPLORATION
\end{enumerate}

\hypertarget{data-acquisition}{%
\subsubsection{Data acquisition}\label{data-acquisition}}

First, we need to explore our given data set. I have published the
original data sets in my github account

\hypertarget{read-data}{%
\paragraph{Read Data}\label{read-data}}

Here, we read the dataset and shorten the feature names for better
readibility in visualizations.

\begin{Shaded}
\begin{Highlighting}[]
\NormalTok{df <-}\StringTok{ }\KeywordTok{read.csv}\NormalTok{(}\StringTok{"https://raw.githubusercontent.com/mkivenson/Business-Analytics-Data-Mining/master/Moneyball%20Regression/moneyball-training-data.csv"}\NormalTok{)[}\OperatorTok{-}\DecValTok{1}\NormalTok{]}
\KeywordTok{names}\NormalTok{(df) <-}\StringTok{ }\KeywordTok{sub}\NormalTok{(}\StringTok{"TEAM_"}\NormalTok{, }\StringTok{""}\NormalTok{, }\KeywordTok{names}\NormalTok{(df))}
\KeywordTok{names}\NormalTok{(df) <-}\StringTok{ }\KeywordTok{sub}\NormalTok{(}\StringTok{"BATTING_"}\NormalTok{, }\StringTok{"bt_"}\NormalTok{, }\KeywordTok{names}\NormalTok{(df))}
\KeywordTok{names}\NormalTok{(df) <-}\StringTok{ }\KeywordTok{sub}\NormalTok{(}\StringTok{"BASERUN_"}\NormalTok{, }\StringTok{"br_"}\NormalTok{, }\KeywordTok{names}\NormalTok{(df))}
\KeywordTok{names}\NormalTok{(df) <-}\StringTok{ }\KeywordTok{sub}\NormalTok{(}\StringTok{"FIELDING_"}\NormalTok{, }\StringTok{"fd_"}\NormalTok{, }\KeywordTok{names}\NormalTok{(df))}
\KeywordTok{names}\NormalTok{(df) <-}\StringTok{ }\KeywordTok{sub}\NormalTok{(}\StringTok{"PITCHING_"}\NormalTok{, }\StringTok{"ph_"}\NormalTok{, }\KeywordTok{names}\NormalTok{(df))}
\KeywordTok{names}\NormalTok{(df) <-}\StringTok{ }\KeywordTok{sub}\NormalTok{(}\StringTok{"TARGET_"}\NormalTok{, }\StringTok{""}\NormalTok{, }\KeywordTok{names}\NormalTok{(df))}
\KeywordTok{head}\NormalTok{(df)}
\end{Highlighting}
\end{Shaded}

\begin{verbatim}
##   WINS bt_H bt_2B bt_3B bt_HR bt_BB bt_SO br_SB br_CS bt_HBP ph_H ph_HR
## 1   39 1445   194    39    13   143   842    NA    NA     NA 9364    84
## 2   70 1339   219    22   190   685  1075    37    28     NA 1347   191
## 3   86 1377   232    35   137   602   917    46    27     NA 1377   137
## 4   70 1387   209    38    96   451   922    43    30     NA 1396    97
## 5   82 1297   186    27   102   472   920    49    39     NA 1297   102
## 6   75 1279   200    36    92   443   973   107    59     NA 1279    92
##   ph_BB ph_SO fd_E fd_DP
## 1   927  5456 1011    NA
## 2   689  1082  193   155
## 3   602   917  175   153
## 4   454   928  164   156
## 5   472   920  138   168
## 6   443   973  123   149
\end{verbatim}

\hypertarget{summary}{%
\paragraph{Summary}\label{summary}}

First, we take a look at a summary of the data. A few things of interest
are revealed:

\begin{itemize}
\tightlist
\item
  bt\_SO, br\_SB, br\_CS, bt\_HBP, ph\_SO, and fd\_DP have missing
  values
\item
  The max values of ph\_H, ph\_BB, ph\_SO, and fd\_E seem abnormally
  high
\end{itemize}

\begin{Shaded}
\begin{Highlighting}[]
\KeywordTok{summary}\NormalTok{(df)}
\end{Highlighting}
\end{Shaded}

\begin{verbatim}
##       WINS             bt_H          bt_2B           bt_3B       
##  Min.   :  0.00   Min.   : 891   Min.   : 69.0   Min.   :  0.00  
##  1st Qu.: 71.00   1st Qu.:1383   1st Qu.:208.0   1st Qu.: 34.00  
##  Median : 82.00   Median :1454   Median :238.0   Median : 47.00  
##  Mean   : 80.79   Mean   :1469   Mean   :241.2   Mean   : 55.25  
##  3rd Qu.: 92.00   3rd Qu.:1537   3rd Qu.:273.0   3rd Qu.: 72.00  
##  Max.   :146.00   Max.   :2554   Max.   :458.0   Max.   :223.00  
##                                                                  
##      bt_HR            bt_BB           bt_SO            br_SB      
##  Min.   :  0.00   Min.   :  0.0   Min.   :   0.0   Min.   :  0.0  
##  1st Qu.: 42.00   1st Qu.:451.0   1st Qu.: 548.0   1st Qu.: 66.0  
##  Median :102.00   Median :512.0   Median : 750.0   Median :101.0  
##  Mean   : 99.61   Mean   :501.6   Mean   : 735.6   Mean   :124.8  
##  3rd Qu.:147.00   3rd Qu.:580.0   3rd Qu.: 930.0   3rd Qu.:156.0  
##  Max.   :264.00   Max.   :878.0   Max.   :1399.0   Max.   :697.0  
##                                   NA's   :102      NA's   :131    
##      br_CS           bt_HBP           ph_H           ph_HR      
##  Min.   :  0.0   Min.   :29.00   Min.   : 1137   Min.   :  0.0  
##  1st Qu.: 38.0   1st Qu.:50.50   1st Qu.: 1419   1st Qu.: 50.0  
##  Median : 49.0   Median :58.00   Median : 1518   Median :107.0  
##  Mean   : 52.8   Mean   :59.36   Mean   : 1779   Mean   :105.7  
##  3rd Qu.: 62.0   3rd Qu.:67.00   3rd Qu.: 1682   3rd Qu.:150.0  
##  Max.   :201.0   Max.   :95.00   Max.   :30132   Max.   :343.0  
##  NA's   :772     NA's   :2085                                   
##      ph_BB            ph_SO              fd_E            fd_DP      
##  Min.   :   0.0   Min.   :    0.0   Min.   :  65.0   Min.   : 52.0  
##  1st Qu.: 476.0   1st Qu.:  615.0   1st Qu.: 127.0   1st Qu.:131.0  
##  Median : 536.5   Median :  813.5   Median : 159.0   Median :149.0  
##  Mean   : 553.0   Mean   :  817.7   Mean   : 246.5   Mean   :146.4  
##  3rd Qu.: 611.0   3rd Qu.:  968.0   3rd Qu.: 249.2   3rd Qu.:164.0  
##  Max.   :3645.0   Max.   :19278.0   Max.   :1898.0   Max.   :228.0  
##                   NA's   :102                        NA's   :286
\end{verbatim}

\hypertarget{dimensions}{%
\paragraph{Dimensions}\label{dimensions}}

Let's see the dimensions of our moneyball training data set.

\begin{Shaded}
\begin{Highlighting}[]
\KeywordTok{dim}\NormalTok{(df)}
\end{Highlighting}
\end{Shaded}

\begin{verbatim}
## [1] 2276   16
\end{verbatim}

\begin{center}\rule{0.5\linewidth}{\linethickness}\end{center}

The training data has 17 columns and 2,276 rows.

The explanatory columns are broken down into four categories:

\begin{itemize}
\tightlist
\item
  Batting
\item
  Base run
\item
  Pitching
\item
  Fielding
\end{itemize}

Below you will see a preview of the columns and the first few
observations broken down into these four categories.

\hypertarget{histogram}{%
\paragraph{Histogram}\label{histogram}}

Next, we create histograms of each of the features and target variable.

\begin{itemize}
\tightlist
\item
  bt\_H, bt\_2B, bt\_BB, br\_CS, bt\_HBP, fd\_DP, WINS all have normal
  distributions
\item
  ph\_H, ph\_BB, ph\_SO, and fd\_E are highly right-skewed
\end{itemize}

\begin{Shaded}
\begin{Highlighting}[]
\KeywordTok{grid.arrange}\NormalTok{(}\KeywordTok{ggplot}\NormalTok{(df, }\KeywordTok{aes}\NormalTok{(bt_H)) }\OperatorTok{+}\StringTok{ }\KeywordTok{geom_histogram}\NormalTok{(}\DataTypeTok{binwidth =} \DecValTok{30}\NormalTok{),}
             \KeywordTok{ggplot}\NormalTok{(df, }\KeywordTok{aes}\NormalTok{(bt_2B)) }\OperatorTok{+}\StringTok{ }\KeywordTok{geom_histogram}\NormalTok{(}\DataTypeTok{binwidth =} \DecValTok{10}\NormalTok{),}
             \KeywordTok{ggplot}\NormalTok{(df, }\KeywordTok{aes}\NormalTok{(bt_3B)) }\OperatorTok{+}\StringTok{ }\KeywordTok{geom_histogram}\NormalTok{(}\DataTypeTok{binwidth =} \DecValTok{10}\NormalTok{),}
             \KeywordTok{ggplot}\NormalTok{(df, }\KeywordTok{aes}\NormalTok{(bt_HR)) }\OperatorTok{+}\StringTok{ }\KeywordTok{geom_histogram}\NormalTok{(}\DataTypeTok{binwidth =} \DecValTok{10}\NormalTok{),}
             \KeywordTok{ggplot}\NormalTok{(df, }\KeywordTok{aes}\NormalTok{(bt_BB)) }\OperatorTok{+}\StringTok{ }\KeywordTok{geom_histogram}\NormalTok{(}\DataTypeTok{binwidth =} \DecValTok{30}\NormalTok{),}
             \KeywordTok{ggplot}\NormalTok{(df, }\KeywordTok{aes}\NormalTok{(bt_SO)) }\OperatorTok{+}\StringTok{ }\KeywordTok{geom_histogram}\NormalTok{(}\DataTypeTok{binwidth =} \DecValTok{50}\NormalTok{),}
             \KeywordTok{ggplot}\NormalTok{(df, }\KeywordTok{aes}\NormalTok{(br_SB)) }\OperatorTok{+}\StringTok{ }\KeywordTok{geom_histogram}\NormalTok{(}\DataTypeTok{binwidth =} \DecValTok{30}\NormalTok{),}
             \KeywordTok{ggplot}\NormalTok{(df, }\KeywordTok{aes}\NormalTok{(br_CS)) }\OperatorTok{+}\StringTok{ }\KeywordTok{geom_histogram}\NormalTok{(}\DataTypeTok{binwidth =} \DecValTok{10}\NormalTok{),}
             \KeywordTok{ggplot}\NormalTok{(df, }\KeywordTok{aes}\NormalTok{(bt_HBP)) }\OperatorTok{+}\StringTok{ }\KeywordTok{geom_histogram}\NormalTok{(}\DataTypeTok{binwidth =} \DecValTok{3}\NormalTok{),}
             \KeywordTok{ggplot}\NormalTok{(df, }\KeywordTok{aes}\NormalTok{(ph_H)) }\OperatorTok{+}\StringTok{ }\KeywordTok{geom_histogram}\NormalTok{(}\DataTypeTok{binwidth =} \DecValTok{100}\NormalTok{),}
             \KeywordTok{ggplot}\NormalTok{(df, }\KeywordTok{aes}\NormalTok{(ph_HR)) }\OperatorTok{+}\StringTok{ }\KeywordTok{geom_histogram}\NormalTok{(}\DataTypeTok{binwidth =} \DecValTok{10}\NormalTok{),}
             \KeywordTok{ggplot}\NormalTok{(df, }\KeywordTok{aes}\NormalTok{(ph_BB)) }\OperatorTok{+}\StringTok{ }\KeywordTok{geom_histogram}\NormalTok{(}\DataTypeTok{binwidth =} \DecValTok{100}\NormalTok{),}
             \KeywordTok{ggplot}\NormalTok{(df, }\KeywordTok{aes}\NormalTok{(ph_SO)) }\OperatorTok{+}\StringTok{ }\KeywordTok{geom_histogram}\NormalTok{(}\DataTypeTok{binwidth =} \DecValTok{30}\NormalTok{),}
             \KeywordTok{ggplot}\NormalTok{(df, }\KeywordTok{aes}\NormalTok{(fd_E)) }\OperatorTok{+}\StringTok{ }\KeywordTok{geom_histogram}\NormalTok{(}\DataTypeTok{binwidth =} \DecValTok{30}\NormalTok{),}
             \KeywordTok{ggplot}\NormalTok{(df, }\KeywordTok{aes}\NormalTok{(fd_DP)) }\OperatorTok{+}\StringTok{ }\KeywordTok{geom_histogram}\NormalTok{(}\DataTypeTok{binwidth =} \DecValTok{10}\NormalTok{),}
             \KeywordTok{ggplot}\NormalTok{(df, }\KeywordTok{aes}\NormalTok{(WINS)) }\OperatorTok{+}\StringTok{ }\KeywordTok{geom_histogram}\NormalTok{(}\DataTypeTok{binwidth =} \DecValTok{5}\NormalTok{),}
             \DataTypeTok{ncol=}\DecValTok{4}\NormalTok{)}
\end{Highlighting}
\end{Shaded}

\begin{center}\includegraphics{VPatel_Assign1_files/figure-latex/fig1-1} \end{center}

\hypertarget{qq-plots}{%
\paragraph{QQ Plots}\label{qq-plots}}

\begin{itemize}
\tightlist
\item
  Most of the features are not lined up with the theoretical QQ plot,
  however this will be addressed by the models we build.
\end{itemize}

\begin{center}\includegraphics{VPatel_Assign1_files/figure-latex/fig2-1} \end{center}

\hypertarget{boxplot}{%
\paragraph{Boxplot}\label{boxplot}}

\begin{itemize}
\tightlist
\item
  Most of the boxplots shown below reflect a long right tail with many
  outliers.
\end{itemize}

\begin{Shaded}
\begin{Highlighting}[]
\KeywordTok{grid.arrange}\NormalTok{(}\KeywordTok{ggplot}\NormalTok{(df, }\KeywordTok{aes}\NormalTok{(}\DataTypeTok{x =} \StringTok{"bt_H"}\NormalTok{, }\DataTypeTok{y =}\NormalTok{ bt_H))}\OperatorTok{+}\KeywordTok{geom_boxplot}\NormalTok{(),}
             \KeywordTok{ggplot}\NormalTok{(df, }\KeywordTok{aes}\NormalTok{(}\DataTypeTok{x =} \StringTok{"bt_2B"}\NormalTok{, }\DataTypeTok{y =}\NormalTok{ bt_2B))}\OperatorTok{+}\KeywordTok{geom_boxplot}\NormalTok{(),}
             \KeywordTok{ggplot}\NormalTok{(df, }\KeywordTok{aes}\NormalTok{(}\DataTypeTok{x =} \StringTok{"bt_3B"}\NormalTok{, }\DataTypeTok{y =}\NormalTok{ bt_3B))}\OperatorTok{+}\KeywordTok{geom_boxplot}\NormalTok{(),}
             \KeywordTok{ggplot}\NormalTok{(df, }\KeywordTok{aes}\NormalTok{(}\DataTypeTok{x =} \StringTok{"bt_HR"}\NormalTok{, }\DataTypeTok{y =}\NormalTok{ bt_HR))}\OperatorTok{+}\KeywordTok{geom_boxplot}\NormalTok{(),}
             \KeywordTok{ggplot}\NormalTok{(df, }\KeywordTok{aes}\NormalTok{(}\DataTypeTok{x =} \StringTok{"bt_BB"}\NormalTok{, }\DataTypeTok{y =}\NormalTok{ bt_BB))}\OperatorTok{+}\KeywordTok{geom_boxplot}\NormalTok{(),}
             \KeywordTok{ggplot}\NormalTok{(df, }\KeywordTok{aes}\NormalTok{(}\DataTypeTok{x =} \StringTok{"bt_SO"}\NormalTok{, }\DataTypeTok{y =}\NormalTok{ bt_SO))}\OperatorTok{+}\KeywordTok{geom_boxplot}\NormalTok{(),}
             \KeywordTok{ggplot}\NormalTok{(df, }\KeywordTok{aes}\NormalTok{(}\DataTypeTok{x =} \StringTok{"br_SB"}\NormalTok{, }\DataTypeTok{y =}\NormalTok{ br_SB))}\OperatorTok{+}\KeywordTok{geom_boxplot}\NormalTok{(),}
             \KeywordTok{ggplot}\NormalTok{(df, }\KeywordTok{aes}\NormalTok{(}\DataTypeTok{x =} \StringTok{"br_CS"}\NormalTok{, }\DataTypeTok{y =}\NormalTok{ br_CS))}\OperatorTok{+}\KeywordTok{geom_boxplot}\NormalTok{(),}
             \KeywordTok{ggplot}\NormalTok{(df, }\KeywordTok{aes}\NormalTok{(}\DataTypeTok{x =} \StringTok{"bt_HBP"}\NormalTok{, }\DataTypeTok{y =}\NormalTok{ bt_HBP))}\OperatorTok{+}\KeywordTok{geom_boxplot}\NormalTok{(),}
             \KeywordTok{ggplot}\NormalTok{(df, }\KeywordTok{aes}\NormalTok{(}\DataTypeTok{x =} \StringTok{"ph_H"}\NormalTok{, }\DataTypeTok{y =}\NormalTok{ ph_H))}\OperatorTok{+}\KeywordTok{geom_boxplot}\NormalTok{(),}
             \KeywordTok{ggplot}\NormalTok{(df, }\KeywordTok{aes}\NormalTok{(}\DataTypeTok{x =} \StringTok{"ph_HR"}\NormalTok{, }\DataTypeTok{y =}\NormalTok{ ph_HR))}\OperatorTok{+}\KeywordTok{geom_boxplot}\NormalTok{(),}
             \KeywordTok{ggplot}\NormalTok{(df, }\KeywordTok{aes}\NormalTok{(}\DataTypeTok{x =} \StringTok{"ph_BB"}\NormalTok{, }\DataTypeTok{y =}\NormalTok{ ph_BB))}\OperatorTok{+}\KeywordTok{geom_boxplot}\NormalTok{(),}
             \KeywordTok{ggplot}\NormalTok{(df, }\KeywordTok{aes}\NormalTok{(}\DataTypeTok{x =} \StringTok{"ph_SO"}\NormalTok{, }\DataTypeTok{y =}\NormalTok{ ph_SO))}\OperatorTok{+}\KeywordTok{geom_boxplot}\NormalTok{(),}
             \KeywordTok{ggplot}\NormalTok{(df, }\KeywordTok{aes}\NormalTok{(}\DataTypeTok{x =} \StringTok{"fd_E"}\NormalTok{, }\DataTypeTok{y =}\NormalTok{ fd_E))}\OperatorTok{+}\KeywordTok{geom_boxplot}\NormalTok{(),}
             \KeywordTok{ggplot}\NormalTok{(df, }\KeywordTok{aes}\NormalTok{(}\DataTypeTok{x =} \StringTok{"fd_DP"}\NormalTok{, }\DataTypeTok{y =}\NormalTok{ fd_DP))}\OperatorTok{+}\KeywordTok{geom_boxplot}\NormalTok{(),}
             \KeywordTok{ggplot}\NormalTok{(df, }\KeywordTok{aes}\NormalTok{(}\DataTypeTok{x =} \StringTok{"WINS"}\NormalTok{, }\DataTypeTok{y =}\NormalTok{ WINS))}\OperatorTok{+}\KeywordTok{geom_boxplot}\NormalTok{(),}
             \DataTypeTok{ncol=}\DecValTok{4}\NormalTok{)}
\end{Highlighting}
\end{Shaded}

\begin{center}\includegraphics{VPatel_Assign1_files/figure-latex/fig3-1} \end{center}

\hypertarget{correlation-plot}{%
\paragraph{Correlation Plot}\label{correlation-plot}}

\begin{itemize}
\tightlist
\item
  There is a strong positive correlation between ph\_H and bt\_H
\item
  There is a strong positive correlation between ph\_HR and bt\_HR
\item
  There is a strong positive correlation between ph\_BB and bt\_BB
\item
  There is a strong positive correlation between ph\_SO and bt\_SO
\item
  There seems to be a weak correlation between bt\_HBP/br\_SB and Wins
\end{itemize}

\begin{Shaded}
\begin{Highlighting}[]
\KeywordTok{corrplot}\NormalTok{(}\KeywordTok{cor}\NormalTok{(df, }\DataTypeTok{use =} \StringTok{"complete.obs"}\NormalTok{), }\DataTypeTok{method =}\StringTok{"color"}\NormalTok{, }\DataTypeTok{type=}\StringTok{"lower"}\NormalTok{, }\DataTypeTok{addrect =} \DecValTok{1}\NormalTok{, }\DataTypeTok{number.cex =} \FloatTok{0.5}\NormalTok{, }\DataTypeTok{sig.level =} \FloatTok{0.30}\NormalTok{,}
         \DataTypeTok{addCoef.col =} \StringTok{"black"}\NormalTok{, }\CommentTok{# Add coefficient of correlation}
         \DataTypeTok{tl.srt =} \DecValTok{25}\NormalTok{, }\CommentTok{# Text label color and rotation}
         \DataTypeTok{tl.cex =} \FloatTok{0.7}\NormalTok{,}
         \DataTypeTok{diag =} \OtherTok{TRUE}\NormalTok{)}
\end{Highlighting}
\end{Shaded}

\begin{center}\includegraphics{VPatel_Assign1_files/figure-latex/fig4-1} \end{center}

\hypertarget{scatter-plots}{%
\paragraph{Scatter Plots}\label{scatter-plots}}

Here, we see a scatter plot of each of the feature variables with the
target variable.

\begin{Shaded}
\begin{Highlighting}[]
\KeywordTok{grid.arrange}\NormalTok{(}\KeywordTok{ggplot}\NormalTok{(df, }\KeywordTok{aes}\NormalTok{(bt_H, WINS)) }\OperatorTok{+}\StringTok{ }\KeywordTok{geom_point}\NormalTok{(}\DataTypeTok{alpha =} \DecValTok{1}\OperatorTok{/}\DecValTok{10}\NormalTok{)}\OperatorTok{+}\KeywordTok{geom_smooth}\NormalTok{(}\DataTypeTok{method=}\NormalTok{lm),}
             \KeywordTok{ggplot}\NormalTok{(df, }\KeywordTok{aes}\NormalTok{(bt_2B, WINS)) }\OperatorTok{+}\StringTok{ }\KeywordTok{geom_point}\NormalTok{(}\DataTypeTok{alpha =} \DecValTok{1}\OperatorTok{/}\DecValTok{10}\NormalTok{)}\OperatorTok{+}\KeywordTok{geom_smooth}\NormalTok{(}\DataTypeTok{method=}\NormalTok{lm),}
             \KeywordTok{ggplot}\NormalTok{(df, }\KeywordTok{aes}\NormalTok{(bt_3B, WINS)) }\OperatorTok{+}\StringTok{ }\KeywordTok{geom_point}\NormalTok{(}\DataTypeTok{alpha =} \DecValTok{1}\OperatorTok{/}\DecValTok{10}\NormalTok{)}\OperatorTok{+}\KeywordTok{geom_smooth}\NormalTok{(}\DataTypeTok{method=}\NormalTok{lm),}
             \KeywordTok{ggplot}\NormalTok{(df, }\KeywordTok{aes}\NormalTok{(bt_HR, WINS)) }\OperatorTok{+}\StringTok{ }\KeywordTok{geom_point}\NormalTok{(}\DataTypeTok{alpha =} \DecValTok{1}\OperatorTok{/}\DecValTok{10}\NormalTok{)}\OperatorTok{+}\KeywordTok{geom_smooth}\NormalTok{(}\DataTypeTok{method=}\NormalTok{lm),}
             \KeywordTok{ggplot}\NormalTok{(df, }\KeywordTok{aes}\NormalTok{(bt_BB, WINS)) }\OperatorTok{+}\StringTok{ }\KeywordTok{geom_point}\NormalTok{(}\DataTypeTok{alpha =} \DecValTok{1}\OperatorTok{/}\DecValTok{10}\NormalTok{)}\OperatorTok{+}\KeywordTok{geom_smooth}\NormalTok{(}\DataTypeTok{method=}\NormalTok{lm),}
             \KeywordTok{ggplot}\NormalTok{(df, }\KeywordTok{aes}\NormalTok{(bt_SO, WINS)) }\OperatorTok{+}\StringTok{ }\KeywordTok{geom_point}\NormalTok{(}\DataTypeTok{alpha =} \DecValTok{1}\OperatorTok{/}\DecValTok{10}\NormalTok{)}\OperatorTok{+}\KeywordTok{geom_smooth}\NormalTok{(}\DataTypeTok{method=}\NormalTok{lm),}
             \KeywordTok{ggplot}\NormalTok{(df, }\KeywordTok{aes}\NormalTok{(br_SB, WINS)) }\OperatorTok{+}\StringTok{ }\KeywordTok{geom_point}\NormalTok{(}\DataTypeTok{alpha =} \DecValTok{1}\OperatorTok{/}\DecValTok{10}\NormalTok{)}\OperatorTok{+}\KeywordTok{geom_smooth}\NormalTok{(}\DataTypeTok{method=}\NormalTok{lm),}
             \KeywordTok{ggplot}\NormalTok{(df, }\KeywordTok{aes}\NormalTok{(br_CS, WINS)) }\OperatorTok{+}\StringTok{ }\KeywordTok{geom_point}\NormalTok{(}\DataTypeTok{alpha =} \DecValTok{1}\OperatorTok{/}\DecValTok{10}\NormalTok{)}\OperatorTok{+}\KeywordTok{geom_smooth}\NormalTok{(}\DataTypeTok{method=}\NormalTok{lm),}
             \KeywordTok{ggplot}\NormalTok{(df, }\KeywordTok{aes}\NormalTok{(bt_HBP, WINS)) }\OperatorTok{+}\StringTok{ }\KeywordTok{geom_point}\NormalTok{(}\DataTypeTok{alpha =} \DecValTok{1}\OperatorTok{/}\DecValTok{10}\NormalTok{)}\OperatorTok{+}\KeywordTok{geom_smooth}\NormalTok{(}\DataTypeTok{method=}\NormalTok{lm),}
             \KeywordTok{ggplot}\NormalTok{(df, }\KeywordTok{aes}\NormalTok{(ph_H, WINS)) }\OperatorTok{+}\StringTok{ }\KeywordTok{geom_point}\NormalTok{(}\DataTypeTok{alpha =} \DecValTok{1}\OperatorTok{/}\DecValTok{10}\NormalTok{)}\OperatorTok{+}\KeywordTok{geom_smooth}\NormalTok{(}\DataTypeTok{method=}\NormalTok{lm),}
             \KeywordTok{ggplot}\NormalTok{(df, }\KeywordTok{aes}\NormalTok{(ph_HR, WINS)) }\OperatorTok{+}\StringTok{ }\KeywordTok{geom_point}\NormalTok{(}\DataTypeTok{alpha =} \DecValTok{1}\OperatorTok{/}\DecValTok{10}\NormalTok{)}\OperatorTok{+}\KeywordTok{geom_smooth}\NormalTok{(}\DataTypeTok{method=}\NormalTok{lm),}
             \KeywordTok{ggplot}\NormalTok{(df, }\KeywordTok{aes}\NormalTok{(ph_BB, WINS)) }\OperatorTok{+}\StringTok{ }\KeywordTok{geom_point}\NormalTok{(}\DataTypeTok{alpha =} \DecValTok{1}\OperatorTok{/}\DecValTok{10}\NormalTok{)}\OperatorTok{+}\KeywordTok{geom_smooth}\NormalTok{(}\DataTypeTok{method=}\NormalTok{lm),}
             \KeywordTok{ggplot}\NormalTok{(df, }\KeywordTok{aes}\NormalTok{(ph_SO, WINS)) }\OperatorTok{+}\StringTok{ }\KeywordTok{geom_point}\NormalTok{(}\DataTypeTok{alpha =} \DecValTok{1}\OperatorTok{/}\DecValTok{10}\NormalTok{)}\OperatorTok{+}\KeywordTok{geom_smooth}\NormalTok{(}\DataTypeTok{method=}\NormalTok{lm),}
             \KeywordTok{ggplot}\NormalTok{(df, }\KeywordTok{aes}\NormalTok{(fd_E, WINS)) }\OperatorTok{+}\StringTok{ }\KeywordTok{geom_point}\NormalTok{(}\DataTypeTok{alpha =} \DecValTok{1}\OperatorTok{/}\DecValTok{10}\NormalTok{)}\OperatorTok{+}\KeywordTok{geom_smooth}\NormalTok{(}\DataTypeTok{method=}\NormalTok{lm),}
             \KeywordTok{ggplot}\NormalTok{(df, }\KeywordTok{aes}\NormalTok{(fd_DP, WINS)) }\OperatorTok{+}\StringTok{ }\KeywordTok{geom_point}\NormalTok{(}\DataTypeTok{alpha =} \DecValTok{1}\OperatorTok{/}\DecValTok{10}\NormalTok{)}\OperatorTok{+}\KeywordTok{geom_smooth}\NormalTok{(}\DataTypeTok{method=}\NormalTok{lm),}
             \DataTypeTok{ncol=}\DecValTok{4}\NormalTok{)}
\end{Highlighting}
\end{Shaded}

\begin{center}\includegraphics{VPatel_Assign1_files/figure-latex/fig5-1} \end{center}

\begin{enumerate}
\def\labelenumi{\arabic{enumi}.}
\setcounter{enumi}{1}
\tightlist
\item
  Data Preparation
\end{enumerate}

\hypertarget{outliers}{%
\subsubsection{Outliers}\label{outliers}}

\hypertarget{extreme-values}{%
\paragraph{Extreme Values}\label{extreme-values}}

While exploring the data, we noticed that the max values of ph\_H,
ph\_BB, ph\_SO, and fd\_E seem abnormally high.

We see that the record for most hits in a season by team (ph\_H) was set
at 1,724 in 1921. However, we also know that the datapoints were
normalized for 162 games in a season. To take a moderate approach, we
will remove the some of the most egggregious outliers that are seen in
these variables.

\begin{Shaded}
\begin{Highlighting}[]
\KeywordTok{grid.arrange}\NormalTok{(}\KeywordTok{ggplot}\NormalTok{(df, }\KeywordTok{aes}\NormalTok{(}\DataTypeTok{x =} \StringTok{"ph_H"}\NormalTok{, }\DataTypeTok{y =}\NormalTok{ ph_H))}\OperatorTok{+}\KeywordTok{geom_boxplot}\NormalTok{(),}
             \KeywordTok{ggplot}\NormalTok{(df, }\KeywordTok{aes}\NormalTok{(}\DataTypeTok{x =} \StringTok{"ph_BB"}\NormalTok{, }\DataTypeTok{y =}\NormalTok{ ph_BB))}\OperatorTok{+}\KeywordTok{geom_boxplot}\NormalTok{(),}
             \KeywordTok{ggplot}\NormalTok{(df, }\KeywordTok{aes}\NormalTok{(}\DataTypeTok{x =} \StringTok{"ph_SO"}\NormalTok{, }\DataTypeTok{y =}\NormalTok{ ph_SO))}\OperatorTok{+}\KeywordTok{geom_boxplot}\NormalTok{(),}
             \KeywordTok{ggplot}\NormalTok{(df, }\KeywordTok{aes}\NormalTok{(}\DataTypeTok{x =} \StringTok{"fd_E"}\NormalTok{, }\DataTypeTok{y =}\NormalTok{ fd_E))}\OperatorTok{+}\KeywordTok{geom_boxplot}\NormalTok{(),}
             \DataTypeTok{ncol=}\DecValTok{4}\NormalTok{)}
\end{Highlighting}
\end{Shaded}

\begin{center}\includegraphics{VPatel_Assign1_files/figure-latex/fig6-1} \end{center}

\begin{Shaded}
\begin{Highlighting}[]
\NormalTok{df <-}\StringTok{ }\KeywordTok{filter}\NormalTok{(df, ph_H }\OperatorTok{<}\StringTok{ }\DecValTok{15000} \OperatorTok{|}\StringTok{ }\NormalTok{ph_BB }\OperatorTok{<}\StringTok{ }\DecValTok{1500} \OperatorTok{|}\StringTok{ }\NormalTok{ph_SO }\OperatorTok{<}\StringTok{ }\DecValTok{3000} \OperatorTok{|}\StringTok{ }\NormalTok{fd_E }\OperatorTok{<}\StringTok{ }\DecValTok{1500}\NormalTok{)}
\end{Highlighting}
\end{Shaded}

\hypertarget{cooks-distance}{%
\paragraph{Cooks Distance}\label{cooks-distance}}

We will also remove influencial outliers using Cooks distance.

\begin{Shaded}
\begin{Highlighting}[]
\NormalTok{mod <-}\StringTok{ }\KeywordTok{lm}\NormalTok{(WINS }\OperatorTok{~}\StringTok{ }\NormalTok{., }\DataTypeTok{data=}\NormalTok{df)}
\NormalTok{cooksd <-}\StringTok{ }\KeywordTok{cooks.distance}\NormalTok{(mod)}
\KeywordTok{plot}\NormalTok{(cooksd, }\DataTypeTok{pch=}\StringTok{"*"}\NormalTok{, }\DataTypeTok{cex=}\DecValTok{2}\NormalTok{, }\DataTypeTok{main=}\StringTok{"Influential Outliers by Cooks distance"}\NormalTok{)}
\KeywordTok{abline}\NormalTok{(}\DataTypeTok{h =} \DecValTok{4}\OperatorTok{*}\KeywordTok{mean}\NormalTok{(cooksd, }\DataTypeTok{na.rm=}\NormalTok{T), }\DataTypeTok{col=}\StringTok{"red"}\NormalTok{)  }\CommentTok{# add cutoff line}
\KeywordTok{text}\NormalTok{(}\DataTypeTok{x=}\DecValTok{1}\OperatorTok{:}\KeywordTok{length}\NormalTok{(cooksd)}\OperatorTok{+}\DecValTok{1}\NormalTok{, }\DataTypeTok{y=}\NormalTok{cooksd, }\DataTypeTok{labels=}\KeywordTok{ifelse}\NormalTok{(cooksd}\OperatorTok{>}\DecValTok{4}\OperatorTok{*}\KeywordTok{mean}\NormalTok{(cooksd, }\DataTypeTok{na.rm=}\NormalTok{T),}\KeywordTok{names}\NormalTok{(cooksd),}\StringTok{""}\NormalTok{), }\DataTypeTok{col=}\StringTok{"red"}\NormalTok{)  }\CommentTok{# add labels}
\end{Highlighting}
\end{Shaded}

\begin{center}\includegraphics{VPatel_Assign1_files/figure-latex/fig7-1} \end{center}

We remove the influencial outliers.

\begin{Shaded}
\begin{Highlighting}[]
\NormalTok{influential <-}\StringTok{ }\KeywordTok{as.numeric}\NormalTok{(}\KeywordTok{names}\NormalTok{(cooksd)[(cooksd }\OperatorTok{>}\StringTok{ }\DecValTok{4}\OperatorTok{*}\KeywordTok{mean}\NormalTok{(cooksd, }\DataTypeTok{na.rm=}\NormalTok{T))])}
\NormalTok{df <-}\StringTok{ }\NormalTok{df[}\OperatorTok{-}\NormalTok{influential, ]}
\end{Highlighting}
\end{Shaded}

\hypertarget{fill-missing-values}{%
\subsubsection{Fill Missing Values}\label{fill-missing-values}}

The following features have missing values.

\begin{itemize}
\tightlist
\item
  bt\_SO - Strikeouts by batters
\item
  br\_SB - Stolen bases
\item
  br\_CS - Caught stealing
\item
  bt\_HBP - Batters hit by pitch (get a free base)
\item
  ph\_SO - Strikeouts by pitchers
\item
  fd\_DP - Double Plays
\end{itemize}

Since most values in bt\_HBP are missing (90\%), we will drop this
feature.

\hypertarget{multivariate-imputation-by-chained-equations-mice}{%
\paragraph{Multivariate Imputation by Chained Equations
(mice)}\label{multivariate-imputation-by-chained-equations-mice}}

We will use Multivariable Imputation by Chained Equations (mice) to fill
the missing variables.

\hypertarget{address-correlated-features}{%
\subsubsection{Address Correlated
Features}\label{address-correlated-features}}

While exploring the data, we noticed several features had strong
positive linear relationships.

Let's run a Variance Inflation Factor test to detect multicollinearity.
Features with a VIF score \textgreater{} 10 will be reviewed.

\begin{Shaded}
\begin{Highlighting}[]
\NormalTok{model1 <-}\StringTok{ }\KeywordTok{lm}\NormalTok{(WINS }\OperatorTok{~}\NormalTok{., }\DataTypeTok{data =}\NormalTok{ df)}
\NormalTok{car}\OperatorTok{::}\KeywordTok{vif}\NormalTok{(model1)}
\end{Highlighting}
\end{Shaded}

\begin{verbatim}
##      bt_H     bt_2B     bt_3B     bt_HR     bt_BB     bt_SO     br_SB 
##  3.820596  2.467157  2.989892 36.501400  6.787771  5.279911  3.862460 
##     br_CS      ph_H     ph_HR     ph_BB     ph_SO      fd_E     fd_DP 
##  3.793169  4.073762 29.596294  6.468847  3.369127  4.988328  1.902235
\end{verbatim}

Let's make another correlation plot with only these features.

\begin{itemize}
\tightlist
\item
  bt\_SO (strikeouts by batters) and bt\_H (base hits by batters) have a
  strong positive correlation
\item
  bt\_H (base hits by batters) and bt\_BB (walks by batters) have a
  strong positive correlation
\item
  ph\_BB (walks allowed) and bt\_BB (walks by batters) have a strong
  negative correlation
\item
  ph\_SO (strikeouts by pitchers) and bt\_SO (strikeouts by batters)
  have a moderate negative correlation
\item
  ph\_HR (homeruns allowed) and bt\_HR (homeruns by batters) have a
  strong negative correlation
\item
  ph\_SO (strikeouts by pitchers) and ph\_BB (walks allowed) have a
  moderate negative correation
\end{itemize}

\begin{Shaded}
\begin{Highlighting}[]
\KeywordTok{corrplot}\NormalTok{(}\KeywordTok{cor}\NormalTok{(}\KeywordTok{subset}\NormalTok{(df, }\DataTypeTok{select =} \KeywordTok{c}\NormalTok{(WINS, bt_H, bt_HR, bt_BB, bt_SO, ph_H, ph_HR, ph_BB, ph_SO)), }\DataTypeTok{use =} \StringTok{"complete.obs"}\NormalTok{), }\DataTypeTok{method =}\StringTok{"color"}\NormalTok{, }\DataTypeTok{type=}\StringTok{"lower"}\NormalTok{, }\DataTypeTok{addrect =} \DecValTok{1}\NormalTok{, }\DataTypeTok{number.cex =} \FloatTok{0.5}\NormalTok{, }\DataTypeTok{sig.level =} \FloatTok{0.30}\NormalTok{,}
         \DataTypeTok{addCoef.col =} \StringTok{"black"}\NormalTok{, }\CommentTok{# Add coefficient of correlation}
         \DataTypeTok{tl.srt =} \DecValTok{25}\NormalTok{, }\CommentTok{# Text label color and rotation}
         \DataTypeTok{tl.cex =} \FloatTok{0.7}\NormalTok{,}
         \DataTypeTok{diag =} \OtherTok{TRUE}\NormalTok{)}
\end{Highlighting}
\end{Shaded}

\begin{center}\includegraphics{VPatel_Assign1_files/figure-latex/fig9-1} \end{center}

To fix this, we can remove some correlated features and combine others.

\begin{itemize}
\tightlist
\item
  Remove bt\_HR. It has an extremely strong correlation with ph\_HR.
\item
  Remove bt\_SO. It has an extremely strong correlation with ph\_SO.
\item
  Replace bt\_H (total base hits by batters) with BT\_1B = bt\_H -
  BT\_2B - BT\_3B - BT\_HR (1B base hits)
\item
  Replace ph\_BB and bt\_BB as a ratio of walks by batters to walks
  allowed
\end{itemize}

\begin{Shaded}
\begin{Highlighting}[]
\NormalTok{df}\OperatorTok{$}\NormalTok{bt_1B <-}\StringTok{ }\NormalTok{df}\OperatorTok{$}\NormalTok{bt_H }\OperatorTok{-}\StringTok{ }\NormalTok{df}\OperatorTok{$}\NormalTok{bt_2B }\OperatorTok{-}\StringTok{ }\NormalTok{df}\OperatorTok{$}\NormalTok{bt_3B }\OperatorTok{-}\StringTok{ }\NormalTok{df}\OperatorTok{$}\NormalTok{bt_HR}
\NormalTok{df}\OperatorTok{$}\NormalTok{BB <-}\StringTok{ }\NormalTok{df}\OperatorTok{$}\NormalTok{bt_BB }\OperatorTok{/}\StringTok{ }\NormalTok{df}\OperatorTok{$}\NormalTok{ph_BB}
\NormalTok{df2 <-}\StringTok{ }\KeywordTok{subset}\NormalTok{(df, }\DataTypeTok{select =} \OperatorTok{-}\KeywordTok{c}\NormalTok{(bt_HR, bt_SO, bt_H, bt_BB, ph_BB))}
\end{Highlighting}
\end{Shaded}

These adjustments result in less multicollinearity.

\begin{Shaded}
\begin{Highlighting}[]
\NormalTok{model1 <-}\StringTok{ }\KeywordTok{lm}\NormalTok{(WINS }\OperatorTok{~}\NormalTok{., }\DataTypeTok{data =}\NormalTok{ df2)}
\NormalTok{car}\OperatorTok{::}\KeywordTok{vif}\NormalTok{(model1)}
\end{Highlighting}
\end{Shaded}

\begin{verbatim}
##    bt_2B    bt_3B    br_SB    br_CS     ph_H    ph_HR    ph_SO     fd_E 
## 1.553145 2.338689 3.650821 3.686438 3.628940 2.311793 1.832450 6.805560 
##    fd_DP    bt_1B       BB 
## 1.865776 2.664315 5.725045
\end{verbatim}

\hypertarget{create-output}{%
\subsubsection{Create Output}\label{create-output}}

\begin{Shaded}
\begin{Highlighting}[]
\KeywordTok{write.csv}\NormalTok{(df, }\StringTok{"baseball_output.csv"}\NormalTok{)}
\end{Highlighting}
\end{Shaded}

\hypertarget{linear-model-1.}{%
\subsubsection{Linear Model 1.}\label{linear-model-1.}}

We will begin with all independent variables and use the back
elimination method to eliminate the non-significant ones.

\begin{Shaded}
\begin{Highlighting}[]
\NormalTok{be_lm1 <-}\StringTok{ }\KeywordTok{lm}\NormalTok{(WINS }\OperatorTok{~}\NormalTok{., }\DataTypeTok{data =}\NormalTok{ df)}
\NormalTok{sum_lm1<-}\StringTok{ }\KeywordTok{summary}\NormalTok{(be_lm1)}
\KeywordTok{par}\NormalTok{(}\DataTypeTok{mfrow=}\KeywordTok{c}\NormalTok{(}\DecValTok{2}\NormalTok{,}\DecValTok{2}\NormalTok{))}
\KeywordTok{plot}\NormalTok{(be_lm1)}
\end{Highlighting}
\end{Shaded}

\includegraphics{VPatel_Assign1_files/figure-latex/unnamed-chunk-12-1.pdf}

We will start by eliminating the variables with high p-values and lowest
significance from the model

Let's take a look at the resulting model:

\begin{Shaded}
\begin{Highlighting}[]
\NormalTok{be_lm2 <-}\StringTok{ }\KeywordTok{lm}\NormalTok{(WINS }\OperatorTok{~}\StringTok{ }\NormalTok{bt_H }\OperatorTok{+}\StringTok{ }\NormalTok{bt_BB }\OperatorTok{+}\StringTok{ }\NormalTok{br_SB }\OperatorTok{+}\StringTok{ }\NormalTok{br_CS, }\DataTypeTok{data =}\NormalTok{df)}
\NormalTok{sum_lm2<-}\KeywordTok{summary}\NormalTok{(be_lm2)}
\KeywordTok{par}\NormalTok{(}\DataTypeTok{mfrow=}\KeywordTok{c}\NormalTok{(}\DecValTok{2}\NormalTok{,}\DecValTok{2}\NormalTok{))}
\KeywordTok{plot}\NormalTok{(be_lm2)}
\end{Highlighting}
\end{Shaded}

\includegraphics{VPatel_Assign1_files/figure-latex/unnamed-chunk-13-1.pdf}

\#\#\#Linear Model 2.

This Linear Model will be built using the variables we believe would
have the highest corelation with WINs.

THe following variables will be used: - Base Hits by batters
(1B,2B,3B,HR) - Walks by batters - Stolen bases - Strikeouts by batters

Let's remove the two variables with low significance:

\begin{Shaded}
\begin{Highlighting}[]
\NormalTok{be_lm3 <-}\StringTok{ }\KeywordTok{lm}\NormalTok{(WINS }\OperatorTok{~}\StringTok{ }\NormalTok{bt_H }\OperatorTok{+}\StringTok{ }\NormalTok{bt_2B }\OperatorTok{+}\StringTok{ }\NormalTok{bt_3B }\OperatorTok{+}\StringTok{ }\NormalTok{bt_HR }\OperatorTok{+}\StringTok{ }\NormalTok{bt_BB }\OperatorTok{+}\StringTok{ }\NormalTok{bt_SO, }\DataTypeTok{data =}\NormalTok{df)}
\NormalTok{sum_lm3<-}\KeywordTok{summary}\NormalTok{(be_lm3)}
\KeywordTok{par}\NormalTok{(}\DataTypeTok{mfrow=}\KeywordTok{c}\NormalTok{(}\DecValTok{2}\NormalTok{,}\DecValTok{2}\NormalTok{))}
\KeywordTok{plot}\NormalTok{(be_lm3)}
\end{Highlighting}
\end{Shaded}

\includegraphics{VPatel_Assign1_files/figure-latex/unnamed-chunk-14-1.pdf}

\begin{Shaded}
\begin{Highlighting}[]
\NormalTok{be_lm4 <-}\StringTok{ }\KeywordTok{lm}\NormalTok{(WINS }\OperatorTok{~}\StringTok{ }
\StringTok{              }\KeywordTok{I}\NormalTok{(bt_H }\OperatorTok{+}\StringTok{ }\NormalTok{bt_BB }
                \OperatorTok{-}\StringTok{ }\NormalTok{ph_H }\OperatorTok{-}\StringTok{ }\NormalTok{ph_BB) }\OperatorTok{+}\StringTok{ }
\StringTok{              }\KeywordTok{I}\NormalTok{(bt_HR }\OperatorTok{-}\StringTok{ }\NormalTok{ph_HR) }\OperatorTok{+}\StringTok{ }
\StringTok{              }\KeywordTok{I}\NormalTok{(bt_SO }\OperatorTok{-}\StringTok{ }\NormalTok{ph_SO) }\OperatorTok{+}\StringTok{ }
\StringTok{              }\KeywordTok{I}\NormalTok{(br_SB }\OperatorTok{-}\StringTok{ }\NormalTok{br_CS) }\OperatorTok{+}
\StringTok{              }\NormalTok{fd_E }\OperatorTok{+}\StringTok{ }\NormalTok{fd_DP , df)}
\NormalTok{sum_lm4<-}\KeywordTok{summary}\NormalTok{(be_lm4)}
\KeywordTok{par}\NormalTok{(}\DataTypeTok{mfrow=}\KeywordTok{c}\NormalTok{(}\DecValTok{2}\NormalTok{,}\DecValTok{2}\NormalTok{))}
\KeywordTok{plot}\NormalTok{(be_lm4)}
\end{Highlighting}
\end{Shaded}

\includegraphics{VPatel_Assign1_files/figure-latex/unnamed-chunk-15-1.pdf}

\begin{Shaded}
\begin{Highlighting}[]
\CommentTok{# list of models and model summaries}
\NormalTok{models <-}\StringTok{ }\KeywordTok{list}\NormalTok{(be_lm1, be_lm2,be_lm3,be_lm4)}
\NormalTok{modsums <-}\StringTok{ }\KeywordTok{list}\NormalTok{(sum_lm1, sum_lm2, sum_lm3, sum_lm4)}
\NormalTok{nmod <-}\StringTok{ }\KeywordTok{length}\NormalTok{(modsums)}

\CommentTok{# storage variables}
\NormalTok{nvar <-}\StringTok{ }\KeywordTok{integer}\NormalTok{(nmod)}
\NormalTok{sigma <-}\StringTok{ }\KeywordTok{numeric}\NormalTok{(nmod)}
\NormalTok{rsq <-}\StringTok{ }\KeywordTok{numeric}\NormalTok{(nmod)}
\NormalTok{adj_rsq <-}\StringTok{ }\KeywordTok{numeric}\NormalTok{(nmod)}
\NormalTok{fstat <-}\StringTok{ }\KeywordTok{numeric}\NormalTok{(nmod)}
\NormalTok{fstat_p <-}\StringTok{ }\KeywordTok{numeric}\NormalTok{(nmod)}
\NormalTok{mse <-}\StringTok{ }\KeywordTok{numeric}\NormalTok{(nmod)}
\NormalTok{rmse <-}\StringTok{ }\KeywordTok{numeric}\NormalTok{(nmod)}

\CommentTok{# loop through model summaries}
\ControlFlowTok{for}\NormalTok{ (j }\ControlFlowTok{in} \DecValTok{1}\OperatorTok{:}\NormalTok{nmod) \{}
\NormalTok{    nvar[j] <-}\StringTok{ }\NormalTok{modsums[[j]]}\OperatorTok{$}\NormalTok{df[}\DecValTok{1}\NormalTok{]}
\NormalTok{    sigma[j] <-}\StringTok{ }\NormalTok{modsums[[j]]}\OperatorTok{$}\NormalTok{sigma}
\NormalTok{    rsq[j] <-}\StringTok{ }\NormalTok{modsums[[j]]}\OperatorTok{$}\NormalTok{r.squared}
\NormalTok{    adj_rsq[j] <-}\StringTok{ }\NormalTok{modsums[[j]]}\OperatorTok{$}\NormalTok{adj.r.squared}
\NormalTok{    fstat[j] <-}\StringTok{ }\NormalTok{modsums[[j]]}\OperatorTok{$}\NormalTok{fstatistic[}\DecValTok{1}\NormalTok{]}
\NormalTok{    fstat_p[j] <-}\StringTok{ }\DecValTok{1} \OperatorTok{-}\StringTok{ }\KeywordTok{pf}\NormalTok{(modsums[[j]]}\OperatorTok{$}\NormalTok{fstatistic[}\DecValTok{1}\NormalTok{], modsums[[j]]}\OperatorTok{$}\NormalTok{fstatistic[}\DecValTok{2}\NormalTok{], }
\NormalTok{                         modsums[[j]]}\OperatorTok{$}\NormalTok{fstatistic[}\DecValTok{3}\NormalTok{])}
\NormalTok{    mse[j] <-}\StringTok{ }\KeywordTok{mean}\NormalTok{(modsums[[j]]}\OperatorTok{$}\NormalTok{residuals}\OperatorTok{^}\DecValTok{2}\NormalTok{)}
\NormalTok{    rmse[j] <-}\StringTok{ }\KeywordTok{sqrt}\NormalTok{(mse[j])}
\NormalTok{\}}

\NormalTok{modnames <-}\StringTok{ }\KeywordTok{paste0}\NormalTok{(}\StringTok{"lm"}\NormalTok{, }\KeywordTok{c}\NormalTok{(}\DecValTok{1}\OperatorTok{:}\NormalTok{nmod))}

\CommentTok{# evaluation dataframe}
\NormalTok{eval <-}\StringTok{ }\KeywordTok{data.frame}\NormalTok{(}\DataTypeTok{Model =}\NormalTok{ modnames, }
                   \DataTypeTok{N_Vars =}\NormalTok{ nvar,}
                   \DataTypeTok{Sigma =}\NormalTok{ sigma,}
                   \DataTypeTok{R_Sq =}\NormalTok{ rsq,}
                   \DataTypeTok{Adj_R_Sq =}\NormalTok{ adj_rsq,}
                   \DataTypeTok{F_Stat =}\NormalTok{ fstat,}
                   \DataTypeTok{F_P_Val =}\NormalTok{ fstat_p,}
                   \DataTypeTok{MSE =}\NormalTok{ mse,}
                   \DataTypeTok{RMSE =}\NormalTok{ rmse)}

\KeywordTok{kable}\NormalTok{(eval, }\DataTypeTok{digits =} \DecValTok{3}\NormalTok{, }\DataTypeTok{align =} \StringTok{'c'}\NormalTok{)}
\end{Highlighting}
\end{Shaded}

\begin{longtable}[]{@{}ccccccccc@{}}
\toprule
Model & N\_Vars & Sigma & R\_Sq & Adj\_R\_Sq & F\_Stat & F\_P\_Val & MSE
& RMSE\tabularnewline
\midrule
\endhead
lm1 & 16 & 12.540 & 0.363 & 0.359 & 85.935 & 0 & 156.134 &
12.495\tabularnewline
lm2 & 5 & 13.714 & 0.243 & 0.242 & 182.565 & 0 & 187.659 &
13.699\tabularnewline
lm3 & 7 & 13.785 & 0.236 & 0.234 & 116.894 & 0 & 189.444 &
13.764\tabularnewline
lm4 & 7 & 14.774 & 0.123 & 0.120 & 52.855 & 0 & 217.602 &
14.751\tabularnewline
\bottomrule
\end{longtable}


\end{document}
